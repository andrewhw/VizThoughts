%% $Id: 014-Summary.tex,v 1.1 2005/04/05 15:07:55 andrew Exp 2

\documentclass{beamer}
%\mode<presentation>{\usetheme{PaloAlto}\usecolortheme{crane}}
\mode<presentation>{\usetheme{Warsaw}\usecolortheme{beaver}}


\usepackage[latin1]{inputenc}
\usepackage{helvet}
\usepackage{graphicx}
\usepackage{xspace}
\usepackage{paralist}
\usepackage{amsmath,amssymb,bm}
\usepackage{amsthm}
\usepackage{multirow}
\usepackage{fancybox}
\usepackage{cancel}
\usepackage{alphalph}
\usepackage{listings}
\usepackage{xcolor}
\usepackage{pagecolor}
\usepackage{pdfcomment}
\usepackage{wasysym}
\usepackage{ulem}
\usepackage{svg}
\usepackage[weather]{ifsym}
\usepackage{colortbl}
\usepackage[linesnumbered,ruled,vlined]{algorithm2e}
\usepackage{listings}
\usepackage{hyperref}
\usepackage{fontawesome}
\usepackage{transparent}


%\usepackage{quotes}

%\usepackage{tgadventor}

%\usepackage[T3]{fontenc}
%\usepackage{libertine}
%\renewcommand*\familydefault{\sfdefault}

\usepackage[T1]{fontenc}
\usepackage[default]{gillius}
\usepackage[absolute,overlay]{textpos}

\setlength{\TPHorizModule}{1pt}
\setlength{\TPVertModule}{1pt}


\DeclareMathOperator*{\sign}{sign}
\DeclareMathOperator*{\argmax}{argmax}
\DeclareMathOperator*{\gain}{gain}
\newcommand{\data}{\ensuremath{\text{\it{Data}}}}
\newcommand{\random}{\ensuremath{\text{\it{~random~}}}}
\newcommand{\epoch}{\ensuremath{\text{\it{~epoch~}}}}
\newcommand{\error}{\ensuremath{\epsilon}}
\newcommand{\errValue}{\ensuremath{\epsilon}}
\newcommand{\pattern}{\ensuremath{\text{\it{~pattern~}}}}
\newcommand{\unusedPatterns}{\ensuremath{\mathcal{U}}}
\newcommand{\nTrain}{\ensuremath{N_\mathcal{T}}}
\newcommand{\nHidden}{\ensuremath{N_\mathcal{H}}}
\newcommand{\nOutput}{\ensuremath{N_\mathcal{O}}}
\newcommand{\nInput}{\ensuremath{N_\mathcal{I}}}
\newcommand{\momentum}{\ensuremath{\rho}}
\newcommand{\usingMomentum}{\ensuremath{\rho > 0}}
\newcommand{\DV}{\ensuremath{V^{{\Delta}}}}
\newcommand{\DW}{\ensuremath{W^{{\Delta}}}}
\newcommand{\sigmoid}{\ensuremath{\text{\it sigmoid}}}
\newcommand{\sigError}{\ensuremath{\text{\it sigmoidError}}}


\pagecolor{black}
\color{white}


\definecolor{grey1}{rgb}{0.9,0.9,0.9}
\definecolor{grey2}{rgb}{0.8,0.8,0.8}
\definecolor{grey3}{rgb}{0.7,0.7,0.7}
\definecolor{grey4}{rgb}{0.6,0.6,0.6}
\definecolor{grey5}{rgb}{0.5,0.5,0.5}
\definecolor{grey6}{rgb}{0.4,0.4,0.4}
\definecolor{grey7}{rgb}{0.3,0.3,0.3}
\definecolor{grey8}{rgb}{0.2,0.2,0.2}
%\definecolor{grey2}{rgb}{0.2,0.2.0.2}
\definecolor{darkgrey}{rgb}{0.4,0.4,0.4}
\definecolor{DarkBlue}{rgb}{0.8,0.8,1.0}
\definecolor{darkgreen}{rgb}{0.0,0.45,0.0}
\definecolor{DarkRed}{rgb}{1.0,0.3,0.3}
\definecolor{keyred}{rgb}{.478,0,0}
%\definecolor{darkyellow}{rgb}{1.0,0.9,0.0}
%\definecolor{darkyellow}{rgb}{0.992, 0.788, 0.219} %0.8,0.7,0.0}
%\definecolor{darkyellow}{rgb}{0.949, 0.604, 0.0} %0.8,0.7,0.0}
\definecolor{darkyellow}{rgb}{1.0, 0.45, 0.0} %0.8,0.7,0.0}
\definecolor{jade}{rgb}{0, 0.525, 0.5} %0.8,0.7,0.0}
%\definecolor{orchid}{rgb}{0.18, 0.18, 0.608} %0.8,0.7,0.0}	51,51,178
\definecolor{orchid}{rgb}{0.294, 0, 0.510} %0.8,0.7,0.0}	51,51,178
\definecolor{lightgreen}{rgb}{0.5,1.0,0.0}
\definecolor{burntumber}{rgb}{0.55, 0.2, 0.14} %0.8,0.7,0.0}
\definecolor{darkblue}{rgb}{0.0, 0.0, 0.6} %0.8,0.7,0.0}
\definecolor{lightcyan}{rgb}{0.44,1,1}
\definecolor{lightgrey}{rgb}{0.65,0.65,0.65}




\newcommand{\pc}[1]{\textcolor{orchid}{\texttt{#1}}}
\newcommand{\comment}[1]{\textcolor{lightgrey}{\tcc{#1}}}

\newcommand{\viterbi}[2]{\ensuremath{\text{\it viterbi}_{#1,#2}}}
\newcommand{\backpointers}[2]{\ensuremath{\text{\it backpointers}_{#1,#2}}}
\newcommand{\bestpathprob}{\ensuremath{\text{\it bestpathprobability}}}
\newcommand{\bestpathptr}{\ensuremath{\text{\it bestpathpointer}}}
\newcommand{\bestpathval}{\ensuremath{\text{\it bestpath}}}

\makeatletter
\newcommand*{\fnsymbolsingle}[1]{%
\ensuremath{%
    \ifcase#1%
    \or *%
    \or \dagger
    \or \ddagger
    %\or \mathframetitle
    %\or \mathparagraph
    %\or \triangleup
    \or \triangleleft
    \or \bowtie
    \or \oplus
    %\or \maltese
    \else
    \@ctrerr
    \fi
}%
}
\newalphalph{\fnsymbolwrap}[wrap]{\fnsymbolsingle}{}
\makeatother
\renewcommand\thefootnote{\fnsymbolwrap{\value{footnote}}}

\def\clap#1{\hbox to 0pt{\hss#1\hss}}
\def\mathllap{\mathpalette\mathllapinternal}
\def\mathrlap{\mathpalette\mathrlapinternal}
\def\mathclap{\mathpalette\mathclapinternal}
\def\mathllapinternal#1#2{\llap{$\mathsurround=0pt#1{#2}$}}
\def\mathrlapinternal#1#2{\rlap{$\mathsurround=0pt#1{#2}$}}
\def\mathclapinternal#1#2{\clap{$\mathsurround=0pt#1{#2}$}}


\newcommand{\eg}{\textit{e.g.},\xspace} % some Latin abreviations in italic
\newcommand{\ie}{\textit{i.e.};\xspace}
\newcommand{\aka}{\textit{a.k.a.}\xspace}
\newcommand{\etc}{\textit{etc}.\@\xspace}
\newcommand{\nil}[1]{#1}
\newcommand{\key}[1]{\textcolor{keyred}{{\bf #1}}}
\newcommand{\keytt}[1]{\textcolor{orchid}{{\tt #1}}}
\newcommand{\keyttbox}[1]{\fcolorbox{orchid!16}{orchid!08}{\textcolor{orchid}{{\tt #1}}}}
\newcommand{\base}[1]{\textcolor{orchid}{\ensuremath{\mathbf{\sf #1}}}}
\newcommand{\keygreen}[1]{\textcolor{lightgreen}{\emph{#1}}}
\newcommand{\blue}[1]{\textcolor{darkblue}{#1}}
\newcommand{\kpoint}[1]{\textcolor{magenta}{Key Point: {\it #1}}}
\newcommand{\kq}[1]{\textcolor{burntumber}{$\mathcal{Q}$: \emph{#1}}}
\newcommand{\theref}{\ensuremath{\dot{.~.}~}}
\newcommand{\since}{\ensuremath{\dot{~}.\dot{~}~}}
\newcommand{\vs}{-\textit{vs}-}
\newcommand{\fade}[1]{\textcolor{darkgrey}{#1}}

\newcommand{\prop}[1]{\ensuremath{\mathbf{\sf #1}}}
\newcommand{\limplies}[2]{\ensuremath{#1 \rightarrow #2}}
\newcommand{\truep}{\ensuremath{\text{true}}}
\newcommand{\falsep}{\ensuremath{\text{false}}}

\def\tokinfo{\ensuremath{\mathcal{I}}\xspace}
\def\dataset{\tokD}
\def\targetfunction{\tokF}
\def\classifier{\tokC}

%\def\tokprob{\ensuremath{{\sf P}}\xspace}
%\newcommand{\probability}[1]{\ensuremath{\tokprob\left({#1}\right)}\xspace}
\def\tokprob{\ensuremath{{\sf P}}\xspace}
%\newcommand{\probability}[1]{\ensuremath{\tokprob_{#1}}\xspace}
\newcommand{\probability}[1]{\ensuremath{\tokprob\left({#1}\right)}\xspace}
\newcommand{\probpi}[1]{\ensuremath{\Pi_{#1}}\xspace}
\newcommand{\probvector}[1]{\ensuremath{\bm{P}\left({#1}\right)}\xspace}
\newcommand{\information}[1]{\ensuremath{\tokinfo\left({#1}\right)}\xspace}
\newcommand{\entropy}[1]{\ensuremath{\tokH\left({#1}\right)}\xspace}


\newcommand{\expectation}[1]{\ensuremath{\mathcal{E}\left[{#1}\right]}\xspace}
\newcommand{\variance}[1]{\ensuremath{\text{Var}\left[{#1}\right]}\xspace}

\newcommand{\probAND}[2]{\ensuremath{{#1}\wedge{#2}}}
\newcommand{\probOR}[2]{\ensuremath{{#1}\vee{#2}}}



\newcommand{\event}[1]{\ensuremath{{#1}}\xspace}
\def\eventA{\event{A}}
\def\eventB{\event{B}}
\def\eventC{\event{C}}
\def\eventD{\event{D}}
\def\eventE{\event{E}}
\def\eventF{\event{F}}
\def\eventG{\event{G}}
\def\eventL{\event{L}}
\def\eventO{\event{O}}
\def\eventP{\event{P}}
\def\eventR{\event{R}}
\def\eventS{\event{S}}
\def\eventW{\event{W}}
\def\eventX{\event{X}}
\def\eventY{\event{Y}}

\def\tokC{\ensuremath{\mathcal{C}}\xspace}
\def\tokD{\ensuremath{\mathcal{D}}\xspace}
\def\tokF{\ensuremath{\mathcal{F}}\xspace}
\def\tokS{\ensuremath{\mathcal{S}}\xspace}
\def\tokH{\ensuremath{\mathcal{H}}\xspace}
\def\tokU{\ensuremath{\mathcal{U}}\xspace}





%	----------------------------------------------------------

\title{Communication through Pictures}
\subtitle{Data Visualization}

\author{Dr.~Andrew Hamilton-Wright}
\institute{School of Computer Science\\
University of Guelph\\
\url{https://github.com/andrewhw/VizThoughts}}

%\date{Winter 2005}
\date{\today}

\definecolor{darkblue}{rgb}{0.0,0.0,0.8}

% bibliography will have no number tags
\makeatletter
\def\@biblabel#1{\hspace*{-\labelsep}}
\makeatother 


\bibliographystyle{apalike}


\begin{document}

\setbeamertemplate{sidebar right}{}


\newcommand*\oldmacro{}%
\let\oldmacro\insertshorttitle%
\renewcommand*\insertshorttitle{%
  \oldmacro\hfill%
\insertframenumber\,/\,\inserttotalframenumber}

\begin{frame}
    \titlepage
\end{frame}


%\begin{frame}
%\frametitle{What we have to work with}
%\begin{tabular}{@{}ll@{}}
%\key{Our Goals}&
%\key{Our Tools:}\\
%\begin{minipage}[t]{0.4\textwidth}
%\visible<2->{To clearly \key{discern} relationships in our \key{data}:}
%	\begin{itemize}
%	\item \visible<2->{scale}
%	\item \visible<2->{order}
%	\item \visible<2->{magnitude}
%	\item \visible<2->{ranges}
%	\item \visible<2->{item groupings}
%	\item \visible<2->{structure}
%	\end{itemize}
%\visible<2->{Both \key{similarities} and \mbox{\key{differences}} are important}
%\end{minipage}
%&
%\begin{minipage}[t]{0.6\textwidth}
%%\visible<3->{\mbox{Human~visual~system} ~}
%%\visible<3->{\includegraphics[width=\textwidth]{figures/Human_visual_pathway}}
%\visible<3->{\includegraphics[width=\textwidth]{figures/Ware-Brain}}
%\end{minipage}
%\end{tabular}
%\footnotetext[1]{\textcolor{darkgrey}{Ware, Colin (2008): {\it Visual Thinking for Design}, Morgan Kaufmann. ISBN:~978-0-12-370896-0}}
%\end{frame}

%\begin{frame}
%\frametitle{What we can perceive}
%\begin{itemize}
%\item Why are some differences so much easier to perceive than others?
%\item We can't miss the ones below even if we try!
%\end{itemize}
%\begin{center}
%\includegraphics[width=0.5\textwidth]{figures/Ware-what-stands-out-1}%
%\footnotemark[1]
%\end{center}
%\begin{itemize}
%\item \visible<2->{Recognizing this type of difference is built into our visual
%		system, in the retina itself, and in V1}
%\end{itemize}
%\footnotetext[1]{\textcolor{darkgrey}{Ware, Colin (2008): {\it Visual Thinking for Design}, Morgan Kaufmann. ISBN:~978-0-12-370896-0}}
%\end{frame}
%\end{frame}

\begin{frame}
\frametitle{What we can perceive}
\begin{tabular}{@{}cc}
\begin{minipage}[c]{0.8\textwidth}
\includegraphics[width=\textwidth]{figures/Ware-what-stands-out}
\end{minipage}
&
\begin{minipage}[c]{0.2\textwidth}
%\begin{tabular}{@{}c@{}}
%\includegraphics[width=\textwidth]{figures/Ware-Brain}\\
\rotatebox{90}{%
\begin{minipage}{0.8\textheight}
{\footnotesize{\textcolor{darkgrey}{Colin Ware (2008): {\it Visual Thinking for Design}, Morgan Kaufmann.}}} %\\
{\footnotesize{\textcolor{darkgrey}{\mbox{ISBN:~978-0-12-370896-0}}}}
\end{minipage}}
%\end{tabular}
\end{minipage}
\end{tabular}
\end{frame}

\begin{frame}
\frametitle{What we can perceive - Easy and Hard}
\begin{tabular}{cc}
\includegraphics[width=0.5\textwidth]{figures/Ware-Easy-and-Hard-T}
&
\includegraphics[width=0.5\textwidth]{figures/Ware-Easy-and-Hard-Colour}
\\
\pause
\includegraphics[width=0.5\textwidth]{figures/Ware-Easy-and-Hard-Bracket}
&
\includegraphics[width=0.5\textwidth]{figures/Ware-Easy-and-Hard-Angle}
%\begin{tabular}{|c||c|}
%\hline
%\includegraphics[trim={1cm 6.5cm 12.5cm 13cm}, clip,
%		width=0.5\textwidth]{figures/Ware-Easy-and-Hard}
%		&
%\includegraphics[trim={1cm 6.5cm 12.5cm 13cm}, clip,
%		width=0.5\textwidth]{figures/Ware-Easy-and-Hard}
%		\\
%\hline
%\includegraphics[trim={1cm 14.15cm 12cm 5.5cm}, clip,
%		width=0.5\textwidth]{figures/Ware-Easy-and-Hard}
%		\\
%\hline
\end{tabular}
\footnotetext[1]{\textcolor{darkgrey}{Ware (2008): {\it Visual Thinking for Design}}}
\end{frame}

\begin{frame}
\frametitle{We must design for the perceptual system we have}
\begin{tabular}{c@{\hspace{3em}}c}
\begin{minipage}[c]{0.8\textwidth}
\includegraphics[width=\textwidth]{figures/Ware-Brain}
\end{minipage}&
\begin{minipage}[c]{0.1\textwidth}
\rotatebox{90}{%
\begin{minipage}{0.8\textheight}
\footnotesize{\textcolor{darkgrey}{Ware (2008): {\it Visual Thinking for Design}}}
\end{minipage}}
\end{minipage}
\end{tabular}
\end{frame}
%\begin{frame}
%\includegraphics[width=\textwidth]{figures/Ware-Easy-and-Hard}
%\end{frame}

%\begin{frame}
%\frametitle{What we can perceive}
%\includegraphics[width=\textwidth]{figures/Ware-Processing-in-V1}
%\end{frame}

%\begin{frame}
%\frametitle{What we can perceive}
%\includegraphics[width=\textwidth]{figures/Ware-Visual-Search}
%\end{frame}
%
%\begin{frame}
%\frametitle{What we can perceive}
%\includegraphics[width=\textwidth]{figures/Ware-Visual-Search-solution}
%\end{frame}

\begin{frame}
\frametitle{A graph?}
\includegraphics[ %trim={1cm 0.5cm 0 0}, clip,
			width=\textwidth]{code/anscombe-scatter-all}
\begin{textblock*}{5in}(0.9in,0.55in)
\begin{minipage}{0.75\textwidth}
\visible<2->{\Large No title? What does this even show?\\
Where are the axis labels (and units)?\\}
\end{minipage}
\end{textblock*}
\begin{textblock*}{3.5in}(2.125in,2.5in)
\begin{minipage}{0.75\textwidth}
\visible<2->{\Large This is not really a graph -- it's just some lines and dots!}
\end{minipage}
\end{textblock*}
\begin{textblock*}{3.5in}(1.75in,1.75in)
\begin{minipage}{0.75\textwidth}
\visible<3->{\Huge \key{\faThumbsDown \faThumbsDown \faThumbsDown ~~~ \faThumbsDown \faThumbsDown \faThumbsDown}}
\end{minipage}
\end{textblock*}
\end{frame}

%\begin{frame}
%\frametitle{Anscombe's \underline{four} data sets}
%\includegraphics[width=\textwidth]{code/anscombe-scatter-all}
%\begin{textblock*}{3.5in}(2in,2.25in)
%\begin{minipage}{0.75\textwidth}
%\visible<2->{\huge Four data sets but only one marker? \key{\faThumbsDown \faThumbsDown \faThumbsDown}}
%\end{minipage}
%\end{textblock*}
%\end{frame}

\begin{frame}
\frametitle{Anscombe's data sets - \underline{size} indicates data set}
\includegraphics[width=\textwidth]{code/anscombe-scatter-size}
\begin{textblock*}{3.25in}(2in,0.75in)
\begin{minipage}{0.75\textwidth}
\visible<2->{\Large Size is one of the weaker channels}
\end{minipage}
\end{textblock*}
\begin{textblock*}{3.25in}(2.2in,2in)
\begin{minipage}{0.75\textwidth}
\visible<2->{\Large Easy to compare nearby points, harder at distance}

\visible<2->{\Large Note transparency (``alpha'') for overplotting}
\end{minipage}
\end{textblock*}
\end{frame}

\begin{frame}
\frametitle{Anscombe's data sets - \underline{hue} indicates data set}
\includegraphics[width=\textwidth]{code/anscombe-scatter-hue}
\begin{textblock*}{3.5in}(2in,2.2in)
\begin{minipage}{0.75\textwidth}
\visible<2->{\Large Colour is a stronger channel \dots} \visible<3->{for those who can perceive it!
\key{\faThumbsDown}}

\visible<3->{\Large Note transparency is not as powerful here -- why?}
\end{minipage}
\end{textblock*}
\end{frame}

\begin{frame}
\frametitle{Anscombe's data sets - \underline{hue and style} indicate data set}
\includegraphics[width=\textwidth]{code/anscombe-scatter-hue+style}
\begin{textblock*}{3.25in}(2.2in,2.5in)
\begin{minipage}{0.75\textwidth}
\visible<2->{\Large Not bad, but overplotting makes this \key{harder than it need be}}
\end{minipage}
\end{textblock*}
\end{frame}

%\begin{frame}
%\frametitle{Anscombe's data sets - using {\tt seaborn swarmplot}}
%\includegraphics[width=\textwidth]{code/anscombe-swarm-hue}
%\begin{textblock*}{3.5in}(1.6in,0.25in)
%\rotatebox{15}{
%\begin{minipage}{0.75\textwidth}
%\visible<2->{\Large Overplotting solved!} \\
%\visible<3->{\Large But we have corrupted our data to do it!
%	\key{\faThumbsDown \faThumbsDown}}\\
%	~\\
%	~\\
%	~\\
%	~\\
%\visible<4->{\Large We also lost the ability to plot by style
%	by using the ``{swarmplot}'' tool!
%	\key{\faThumbsDown \faThumbsDown \faThumbsDown \faThumbsDown}}
%\end{minipage}}
%\end{textblock*}
%\end{frame}

\begin{frame}
\frametitle{Anscombe's data sets -- separate plots, one figure}
\begin{center}
%\begin{tabular}{c@{}cc@{}c}
\begin{tabular}{cc}
%\rotatebox{90}{Data Set I}&
\includegraphics[width=0.46\textwidth]{code/anscombe-scatter-dataset-I} &
%\rotatebox{90}{Data Set II}&
\includegraphics[width=0.46\textwidth]{code/anscombe-scatter-dataset-II} \\
%\rotatebox{90}{Data Set III}&
\includegraphics[width=0.46\textwidth]{code/anscombe-scatter-dataset-III} &
%\rotatebox{90}{Data Set IV}&
\includegraphics[width=0.46\textwidth]{code/anscombe-scatter-dataset-IV} \\
\end{tabular}
\end{center}
\begin{textblock*}{4.5in}(1.25in,2in)
\begin{minipage}{0.75\textwidth}
\visible<2->{\Large
Overplotting solved, and each data set is clearly identified! \faThumbsUp}
\visible<2->{\key{But \dots}}
\end{minipage}
\end{textblock*}
\end{frame}

\begin{frame}
\frametitle{Anscombe's data sets -- separate plots, one figure}
\begin{center}
%\begin{tabular}{c@{}cc@{}c}
\begin{tabular}{cc}
%\rotatebox{90}{Data Set I}&
\includegraphics[width=0.46\textwidth]{code/anscombe-scatter-dataset-I} &
%\rotatebox{90}{Data Set II}&
\includegraphics[width=0.46\textwidth]{code/anscombe-scatter-dataset-II} \\
%\rotatebox{90}{Data Set III}&
\includegraphics[width=0.46\textwidth]{code/anscombe-scatter-dataset-III} &
%\rotatebox{90}{Data Set IV}&
\includegraphics[width=0.46\textwidth]{code/anscombe-scatter-dataset-IV} \\
\end{tabular}
\end{center}
\begin{textblock*}{4.5in}(1.25in,2in)
\begin{minipage}{0.75\textwidth}
\visible<1->{\Large \key{Each plot has a separate range!}}
\\
\visible<1->{\Large It is \key{very} hard to compare across plots.
\key{\faThumbsDown \faThumbsDown \faThumbsDown
\faThumbsDown \faThumbsDown \faThumbsDown
\faThumbsDown \faThumbsDown \faThumbsDown}}
\end{minipage}
\end{textblock*}
\end{frame}

\begin{frame}
\frametitle{Anscombe's data sets -- from this\dots}
\begin{center}
%\begin{tabular}{c@{}cc@{}c}
\begin{tabular}{cc}
%\rotatebox{90}{Data Set I}&
\includegraphics[width=0.46\textwidth]{code/anscombe-scatter-dataset-I} &
%\rotatebox{90}{Data Set II}&
\includegraphics[width=0.46\textwidth]{code/anscombe-scatter-dataset-II} \\
%\rotatebox{90}{Data Set III}&
\includegraphics[width=0.46\textwidth]{code/anscombe-scatter-dataset-III} &
%\rotatebox{90}{Data Set IV}&
\includegraphics[width=0.46\textwidth]{code/anscombe-scatter-dataset-IV} \\
\end{tabular}
\end{center}
\begin{textblock*}{1in}(2.5in,1.8in)
\key{\Huge \faThumbsDown}
\end{textblock*}
\end{frame}

\begin{frame}
\frametitle{Anscombe's data sets -- to this! Axes fixed!}
\begin{center}
%\begin{tabular}{c@{}cc@{}c}
\begin{tabular}{cc}
%\rotatebox{90}{Data Set I}&
\includegraphics[width=0.46\textwidth]{code/anscombe-scatter-dataset-I-limits} &
%\rotatebox{90}{Data Set II}&
\includegraphics[width=0.46\textwidth]{code/anscombe-scatter-dataset-II-limits} \\
%\rotatebox{90}{Data Set III}&
\includegraphics[width=0.46\textwidth]{code/anscombe-scatter-dataset-III-limits} &
%\rotatebox{90}{Data Set IV}&
\includegraphics[width=0.46\textwidth]{code/anscombe-scatter-dataset-IV-limits} \\
\end{tabular}
\end{center}
\begin{textblock*}{3.5in}(2.5in,1.8in)
\begin{minipage}{0.75\textwidth}
%\visible<2->{\Huge \key{\faThumbsUp}}
\end{minipage}
\end{textblock*}
\end{frame}


\begin{frame}
\frametitle{Anscombe's data sets - linear model plot}
\begin{tabular}{@{}cc}
\begin{minipage}[c]{0.75\textwidth}
\includegraphics[width=\textwidth]{code/anscombe-lmplot}
\end{minipage}&
\begin{minipage}[c]{0.25\textwidth}
{\scriptsize \tt sns.lmplot(x="x", y="y", col="dataset", hue="dataset", data=df, col\_wrap=2)}
\begin{itemize}
\item \pause common axes?  {\key \faThumbsUp}
\item id data sets?  {\key \faThumbsUp}
\item colour as highlight only {\key \faThumbsUp}
\end{itemize}
\end{minipage}
\end{tabular}
\end{frame}



\begin{frame}
\frametitle{Boxplots}
\begin{tabular}{@{}cc}
\begin{minipage}{0.5\textwidth}
\includegraphics[width=\textwidth]{code/boxplot}
\end{minipage}&
\begin{minipage}{0.5\textwidth}
\visible<3->{\includegraphics[width=\textwidth]{code/boxplot-swarm}}
\end{minipage}
\end{tabular}
\begin{itemize}
\item A boxplot provides a ``5 number summary'': median, lower/upper quartiles, min/max. \pause Are these useful for \key{your} data? 
\item A boxplot will encourage you to think of your data as centrally tended, even when it is not.
\begin{center}
{\large \key{Always plot the points} to see what is going on.}
%(We have ``jiggled'' the data here.)
\end{center}
%\vspace{1em}
{\scriptsize
{\tt sns.boxplot(x="Source", y="Measure", data=df)}\\
{\tt sns.swarmplot(x="Source", y="Measure", data=df, color="0.25")}}
\end{itemize}
\end{frame}

%\begin{frame}
%\frametitle{Boxplots}
%\begin{tabular}{@{}cc}
%\begin{minipage}{0.5\textwidth}
%\includegraphics[width=\textwidth]{code/boxplot}
%\end{minipage}&
%\begin{minipage}{0.5\textwidth}
%\visible<1->{\includegraphics[width=\textwidth]{code/boxplot-swarm}}
%\end{minipage}
%\end{tabular}
%\begin{description}
%\item[A] a normal $\mathcal{N}(\mu=10,\sigma=4)$ distribution
%\item[B] a Laplace
%		$\mathcal{L}(\mu=10,b=2.5)$ distribution
%\item[C] bimodal:
%		$\mathcal{N}(\mu=4,\sigma=2)$ +
%		$\mathcal{N}(\mu=16,\sigma=1)$
%\item[D] two ranges of points,
%		$[-2 \dots 7]$,
%		$[16 \dots 35]$
%\item[E] two ranges of points,
%		$[-1 \dots 10]$,
%		$[2 \dots 13]$
%\item[F] a uniform $\mathcal{U}[2 \dots 20]$ distribution
%\item[G] an exponential $\mathcal{E}(\lambda=0.125)$ distribution ($\lambda = \frac{1}{\mu}$)
%\end{description}
%\end{frame}

\begin{frame}
\frametitle{Boxplots and whiskers}
\begin{tabular}{@{}cc}
\begin{minipage}{0.5\textwidth}
\includegraphics[width=\textwidth]{code/boxplot}
\end{minipage}&
\begin{minipage}{0.5\textwidth}
\includegraphics[width=\textwidth]{code/boxplot-swarm}
\end{minipage}
\end{tabular}
\begin{itemize}
\item Make sure you know where your whiskers go to!
\item
\key{R/matplotlib(seaborn)}: box=quartiles, whisker = furthest point within 1.5 of quartile length (on that side)
\item
\key{Excel}: default = whisker to max/min
\item
\key{Others}: various, and commonly 2 standard deviations
%
%boxplots
%display the two central quartiles within the box enclosure,
%and include a whisker that extends to the most extreme data point that
%lies within a distance of 1.5 times the length of the box enclosure.
%In 
%cases where the whiskers do not extend to the min/max value, any outlying values beyond the whisker are shown as outliers (dots).''
\end{itemize}
\begin{center}
\key{Be sure you know what \underline{your} package does!}
\end{center}
\end{frame}

%\begin{frame}
%\frametitle{Boxplots and whiskers}
%\begin{tabular}{@{}cc}
%\begin{minipage}{0.5\textwidth}
%\includegraphics[width=\textwidth]{code/boxplot}
%\end{minipage}&
%\begin{minipage}{0.5\textwidth}
%\includegraphics[width=\textwidth]{code/boxplot-swarm}
%\end{minipage}
%\end{tabular}
%\begin{itemize}
%\item \key{Excel} boxplots \underline{do not} include outliers by default, and the whisker goes to the min/max point.  Boxes showing Q$_2$ and Q$_3$ are unchanged.
%\pause
%\item Other packages may create boxplots where instead of quartiles, the boxes and whiskers are based on (1 \& 2) standard deviations!
%\key{Be sure you understand what your package shows!}
%\end{itemize}
%\end{frame}

\begin{frame}
\frametitle{Scatters and Lines}
\begin{tabular}{@{}cc}
\begin{minipage}{0.5\textwidth}
\includegraphics[width=\textwidth]{code/anscombe-lineplot-nomarkers-despine}
\end{minipage}&
\begin{minipage}{0.5\textwidth}
\visible<2->{\includegraphics[width=\textwidth]{code/anscombe-lineplot-withmarkers-despine}}
\end{minipage}\\
\begin{minipage}{0.5\textwidth}
\visible<2->{\includegraphics[width=\textwidth]{code/anscombe-lineplot-scatter-despine}}
\end{minipage}&
\begin{minipage}{0.5\textwidth}
\begin{itemize}
\item lines make association between points much more visible, {\large but \dots}
\item \visible<2->{lines imply a series -- don't use them if there isn't one}
\item \visible<2->{don't hide your sample points}
\end{itemize}
\end{minipage}\\
\end{tabular}
\end{frame}

\begin{frame}
\frametitle{Scatters and Lines}
\begin{tabular}{@{}cc}
\begin{minipage}{0.5\textwidth}
%\visible<1>
{\includegraphics[width=\textwidth]{code/anscombe-lineplot-nomarkers}}
\end{minipage}&
\begin{minipage}{0.5\textwidth}
\includegraphics[width=\textwidth]{code/anscombe-lineplot-withmarkers}
\end{minipage}\\
\begin{minipage}{0.5\textwidth}
\includegraphics[width=\textwidth]{code/anscombe-lineplot-scatter}
\end{minipage}&
\begin{minipage}{0.5\textwidth}
\begin{itemize}
\item note that the ``data box'' makes the outlying $\times$ point easier to see as it is distinguished from the plot above \dots
%\pause
%\item but it helps hide the data point if the plot is by itself!
\end{itemize}
\end{minipage}\\
\end{tabular}
\end{frame}


\begin{frame}
\frametitle{Categorical Data}
\begin{tabular}{@{}cc}
\begin{minipage}{0.4\textwidth}
\includegraphics[width=\textwidth]{code/barplot-tips}
\end{minipage}&
\begin{minipage}{0.6\textwidth}
\begin{itemize}
\item use barplots for categorical data
\item \visible<2->{works both vertically and horizontally:}
\begin{itemize}
\item \visible<2->{only time you should put a
		\key{dependent variable} on the $x$ axis!}
\end{itemize}
\end{itemize}
\end{minipage}\\
\end{tabular}
\begin{tabular}{@{}cc}
\begin{minipage}[c]{0.75\textwidth}
\visible<3->{Far, far better than pies!}
	\begin{itemize}
	\item \visible<3->{estimating relative size of pie sections is hard -- visual cortex does not extract fine distinctions between angles under rotation}
	\item \visible<3->{but bars are easy to compare!}
	\item \visible<4->{almost impossible to compare \key{two} pies}
	\item \visible<4->{they take up more space for the same data, and need to be much larger to be understood}
	\item \visible<4->{``3D'' effects on pies make this all worse}
	\end{itemize}
\end{minipage}&
\begin{minipage}[c]{0.25\textwidth}
\begin{tabular}{@{}c@{}}
\visible<2->{\includegraphics[width=\textwidth]{code/barplot-tips-horizontal-smoker}}\\
\visible<3->{\includegraphics[width=\textwidth]{code/barplot-tips-horizontal-nosmoker}}\\
\end{tabular}
%\begin{tabular}{@{}c@{\hspace{0.25em}}c@{}}
%\visible<2->{\rotatebox{90}{\scriptsize Smoker}}&\visible<2->{\includegraphics[width=\textwidth]{code/barplot-tips-horizontal-smoker}}\\
%\visible<2->{\rotatebox{90}{\scriptsize ~~~Nonsmoker}}&\visible<2->{\includegraphics[width=\textwidth]{code/barplot-tips-horizontal-nosmoker}}\\
%\end{tabular}
\end{minipage}\\
\end{tabular}
\end{frame}

%\begin{frame}
%\frametitle{When to jiggle?}
%When does it \key{make sense} to jiggle datapoints?
%
%\pause
%You can jiggle points as long as they do not cross columns:
%\begin{itemize}
%\item when your data is \key{categorical} in some dimension
%		(\eg treatment names, locations)
%\item when your data is \key{discrete} -- jiggling on a scatterplot
%		is otherwise not recommended.
%\vspace{1em}
%\end{itemize}
%\end{frame}



\begin{frame}
\frametitle{File format -- raster graphics}
\begin{center}
\includegraphics[width=0.8\textwidth]{code/bateman-PNG-zoom}
\end{center}\vspace{-1em}
PNG, JPEG, GIF, TIFF \etc \pause \key{\faThumbsDown}

These are called ``raster graphic'' formats.  Their data is in \key{pixels}.
They all look like crap when zoomed in.
\key{If you are drawing your own figure, it doesn't have to be like this!}
\end{frame}


\begin{frame}
\frametitle{File format -- vector graphics}
\begin{center}
\includegraphics[width=0.8\textwidth]{code/bateman-PDF-zoom}
\end{center}\vspace{-1em}
\key{PDF}, \key{EPS} \faThumbsUp

These \key{vector graphic} data formats can store ``\key{pen strokes}'' instead of \sout{pixels}, so the image can withstand arbitrary zoom.

\key{Use everywhere}, but \key{especially on {\large POSTERS}!}
\end{frame}


\begin{frame}
\frametitle{Perceptual Design Constraints and Strategies}
\begin{tabular}{@{}cc}
\begin{minipage}{0.65\textwidth}
\begin{itemize}
\item to make something easy to find, make it different from
		surroundings according to a primary visual channel
\item to make several things easy to search, use multiple channels
\end{itemize}
\end{minipage}&
\begin{minipage}{0.4\textwidth}
\includegraphics[width=\textwidth]{figures/Ware-GDP-example}\footnotemark[2]
\end{minipage}
\end{tabular}
\begin{itemize}
\item using two channels for same distinction makes it easier
\item we only have 8-10 channels, and only $\sim 3$ steps per channel
\pause
\item
		\textcolor{red}{c}%
		\textcolor{orange}{o}%
		\textcolor{brown}{l}%
		\textcolor{jade}{o}%
		\textcolor{blue}{u}%
		\textcolor{purple}{r} is powerful but hard to use well:
	\begin{itemize}
	\item common perceptual difference: red/green + blue/green
	%\item \theref we can't depend on colour being visible
	\item
		``\textcolor{jade}{angry}
		\textcolor{cyan}{fruit}
		\textcolor{blue}{salad}''
		\textcolor{brown}{is}
		\textcolor{magenta}{awful}
		\textcolor{orange}{--}
		\textcolor{orchid}{and}
		\underline{\textcolor{pink}{contrast}
		\textcolor{yellow}{is}
		\textcolor{green}{critical}}
	\item projector has less dynamic range than a monitor
	\end{itemize}
	\pause
\item
	\textcolor{grey1}{i}%
	\textcolor{grey2}{n}%
	\textcolor{grey3}{t}%
	\textcolor{grey4}{e}%
	\textcolor{grey5}{n}%
	\textcolor{grey6}{s}%
	\textcolor{grey7}{i}%
	\textcolor{grey8}{t}%
	\textcolor{black}{y}
	is accessible, powerful and easy to use
	\begin{itemize}
	\item combine intensity with colour in gradients
	\item ensure \key{monotonic} change in intensity
	\end{itemize}
\end{itemize}
\footnotetext[2]{\textcolor{darkgrey}{Ware (2008): {\it Visual Thinking for Design}}}
\end{frame}

\begin{frame}
\frametitle{Improving the visualization}
\begin{itemize}
\item reduce the clutter -- improve the ``data-ink ratio''\footnote{\textcolor{darkgrey}{E.~Tufte (1983): {\it The Visual Display of Quantitative Data}, Graphics Press.\\ ISBN:  0-961-3921-2-6}}
\begin{center}
\begin{tabular}{cc}
\includegraphics[width=0.3\textwidth]{figures/Dir1}&
\includegraphics[width=0.3\textwidth]{figures/Dir2}
\end{tabular}
\end{center}
\item choose distinct graphical elements; separable by channel; avoid overlap
\item ensure data series are referenced; labels + legend \emph{must} be clear of the data
\item use all your space; log axes may be appropriate to spread information
\item \key{provide explanations and draw conclusions}
\end{itemize}
\end{frame}

\begin{frame}
\frametitle{Questions?}
\begin{center}
{\Huge Thank-you for your attention!}
\vspace{3em}
\end{center}
\begin{center}
{\Huge ~\faSmileO~}
\vspace{3em}

\url{https://github.com/andrewhw/VizThoughts}
\end{center}
\end{frame}

%\begin{frame}
%\frametitle{Summary}
%\begin{itemize}
%\item think about what you want to say -- attach this to a primary channel
%\item choose a representation that leverages visual channels, but does
%		not overwhelm the viewer with fine distinctions (\eg too many colours, shapes)
%\item use complementary channels for different data
%\item use ink sparingly -- maximize the data-ink ratio
%\item don't let your tools get the better of you!
%\item explain your visualization in your document
%\end{itemize}
%\end{frame}

%\begin{frame}
%\frametitle{File format}
%\begin{tabular}{!{}cc}
%\includegraphics[width=0.5\textwidth]{code/bateman-PNG-zoom}&
%\visible<2->{\includegraphics[width=0.5\textwidth]{code/bateman-PDF-zoom}}\\
%Raster Graphics & \visible<2->{Vector Graphics}\\
%PNG, GIF, JPEG & \visible<2->{PDF, EPS}\\
%\end{tabular}
%\end{frame}

\end{document}
